%% Hello emacs, this is -*- latex -*-
\typeout{ ====================================================================}
\typeout{ This is file published.tex, created at 13-Jun-2004 }
\typeout{ Maintained by Andre Rabello dos Anjos <Andre.dos.Anjos@cern.ch> }
\typeout{ ====================================================================}

% 2006
%\cite{aa:tns-06}
%\cite{aa:jinst-06}
%\cite{aa:chep-06-01}
%\cite{aa:nim-06}

% 2005
%\cite{aa:rt-05}
%\cite{aa:rt-05-01}
%\cite{aa:enfpc-05}
%\cite{aa:enfpc-05a}

% 2004
%\cite{aa:tns-2004}
%\cite{aa:tns-2004-2}
%\cite{aa:tns-2004-3}
%\cite{aa:tns-04-04}
%\cite{aa:tns-04-05}
%\cite{aa:chep-04-01}
%\cite{aa:enfpc-04}

% 2003
%\cite{aa:rt-2003}
%\cite{aa:chep-2003}
%\cite{aa:chep-2003-2}
%\cite{aa:nima-2003}
%\cite{aa:como-2003}
%\cite{aa:chep-2003-3}

% 2001
%\cite{aa:enfpc-01}

% 2000
%\cite{aa:enfpc-00}

% 1999
%\cite{aa:enfpc-99}

% 1998
%\cite{aa:tns-1998}
%\cite{aa:enfpc-98}
%\cite{aa:vecpar-1998}

% 1997
%\cite{aa:acat-1997} *

% 1996
%\cite{aa:cba-1996} % Esse aqui não precisa mencionar...

%Technical notes
%\cite{aa:lvl2-coding}
%\cite{aa:ef-urd}
%\cite{aa:ef-ooad}
%\cite{aa:ttn8-98}
%\cite{aa:ttn17-98}
%\cite{aa:ttn20-98}
%\cite{aa:ttn25-98}

\chapter{Publicações}
\label{ap:published}

Durante a execução deste estudo, um conjunto de trabalhos foi publicado, com a
participação do autor, na forma de artigos e notas técnicas. Este apêndice
traz uma descrição deste material em ordem cronológica reversa. Os artigos
mais recentes incluem a atuação durante o estudo que culmina nesta tese de
doutoramento, enquanto que os mais antigos (anteriores ao ano de 2001), foram
publicados durante a elaboração da dissertação do mestrado e da tese de
graduação.

\paragraph{Agosto de 2006: \eng{Deployment of the ATLAS High-Level
Trigger}} Publicado na \eng{IEEE Transactions on Nuclear
Science}. Referência~\cite{aa:tns-06}.

\begin{quotation}
O teste combinado com feixe do ATLAS na segunda metade de 2004 contou com o
primeiro emprego dos Filtros de Alto-Nível do ATLAS (HLT). Este artigo revê a
experiência adquirida durante este exercício, o estado atual e melhorias que
deverão ser executadas antes do início do experimento. O artigo inclui
discussões sobre o sistema de fluxo de dados, a integração de algoritmos de
seleção, procedimentos de teste, distribuição de \eng{software} e instalação.
\end{quotation}

\paragraph{Maio de 2006: \eng{A configuration system for the ATLAS
trigger}} Publicado no \eng{Journal of
Instrumentation}. Referência~\cite{aa:jinst-06}.

\begin{quotation}
Este artigo descreve o sistema a ser empregado para a configuração dos Filtros
de Alto Nível do experimento ATLAS. Tal sistema provê ao Sistema de Filtragem
todos os parâmetros necessários à tomada de decisões e para a salvaguarda de
seu histórico. O mesmo sistema reconfigura a reconstrução de eventos, as
simulações de Monte Carlo e a análise de dados. Ferramentas para acessar e
manipular este sistema também são discutidas.
\end{quotation}

\paragraph{Abril de 2006: \eng{Neural triggering system operating on
high resolution calorimetry information}} Publicado na \eng{Nuclear
Instruments and Methods in Physics Research}. Referência~\cite{aa:nim-06}.

\begin{quotation}
Este artigo apresenta um sistema de discriminação elétron/jato para operar no
Segundo Nível de Filtragem do experimento ATLAS. De forma a manipular a grande
dimensionalidade de dados, as RoIs são organizadas na forma de anéis
concêntricos de energia e, desta forma, tanto compressão como um melhor
desempenho podem ser obtidos. A informação dos anéis é alimentada em uma rede
neural sem realimentação (\eng{feedforward}). Esta implementação resulta em
uma deteção de elétrons em 97\% contra um falso alarme de apenas 3\%. A cadeia
completa de discriminação é executável em menos de 500~$\mu$s.
\end{quotation}

\paragraph{Fevereiro de 2006: \eng{Testing on a Large Scale : running
the ATLAS Data Acquisition and High Level Trigger Software on 700 PC Nodes}}
Apresentado no \eng{Computing In High Energy and Nuclear
Physics}. Referência~\cite{aa:chep-06-01}.

\begin{quotation}
O Sistema de Filtragem e Aquisição de Dados do ATLAS (TDAQ) terá à sua
disposição, inicialmente, 2000 PCs assuminido as função de controlar, ler e
filtrar os dados dos eventos do experimento. Este grande número de nós de
processamento será adquirido somente em 2007. Em julho de 2005, no entanto, o
CERN disponibilizou o uso de seu sistema de processamento \eng{batch} durante
5 semanas para testes em larga escala dos elementos do TDAQ. No total, 700 nós
de processamento foram utilizados no exercício. Este artigo discute os
diversos aspectos abordados durante o teste, enfatizando um conjunto de
melhorias que devam ser executadas antes da operação final.
\end{quotation}

\paragraph{Outubro de 2005: Discriminação Neural de Elétrons no Segundo Nível
de Trigger do ATLAS} Apresentado no XXVI Encontro Nacional de Física de
Partículas e Campos. Referência \cite{aa:enfpc-05a}. 

\begin{quotation}
Este trabalho apresenta um discriminador neural para o segundo nível de
filtragem do ATLAS, atuando no problema de separação elétron/jato baseado em
informações de calorimetria. Para reduzir a alta dimensionalidade dos dados de
entrada, as regiões de interesse (RoI) identificadas no primeiro nível são
organizadas em anéis concêntricos de deposição energética. Este tipo de
pré-processamento dos dados permite eficiente compactação dos sinais e alcança
elevada capacidade de identificar elétrons. Atualmente, esse sistema vem sendo
portado para o ambiente de emulação do sistema de filtragem ATHENA, de modo a se
obter uma avaliação realística de seu desempenho. O ambiente tem o objetivo de
simular o comportamento do sistema de filtragem, ajudando, desta forma, no
desenvolvimento e validação dos algoritmos. Em caráter comparativo, o sistema
proposto foi também implementado usando a tecnologia DSP.
\end{quotation}

\paragraph{Outubro de 2005: Otimização do Sistema de Trigger do Segundo Nível
do ATLAS Baseado em Calorimetria} Também apresentado no XXVI Encontro Nacional
de Física de Partículas e Campos. Referência \cite{aa:enfpc-05}.

\begin{quotation}
Este trabalho apresenta um discriminador neural que opera sobre as quantidades
calculadas pelo algoritmo T2Calo, responsável pela deteção elétron/jato no
Segundo Nível de Filtragem do experimento ATLAS. Este sistema de deteção
melhora a eficiência de deteção em quase 10 pontos percentuais, mantendo um
nível de desempenho compatível com as restrições operacionais do sistema de
filtragem.
\end{quotation}

\paragraph{Julho de 2005: \eng{Overview of the High-Level Trigger
Electron and Photon Selection for the ATLAS Experiment at the LHC}.}
Apresentado na \eng{14th IEEE - NPSS Real Time Conference 2005 Nuclear Plasma
Sciences Society}. Referência~\cite{aa:rt-05-01}.

\begin{quotation}
Os Altos Níveis de Filtragem do experimento ATLAS estão descritos em
\eng{software}, ao contrário de seu Primeiro Nível, que é baseado em
\eng{hardware}. Estes sistemas deverão reduzir a taxa de eventos de entrada de
$~$75~kHz na saída do Primeiro Nível de Filtragem para apenas $~$200~Hz na
saída do Filtro de Eventos. Os eventos selecionados serão finalmente guardados
definitivamente em mídia permamente. Este artigo descreve uma visão geral de
tal sistema. O desempenho é avaliado segundo simulações de Monte Carlo, tendo
sido parcialmente demonstrado durante a bancada de testes com feixe em 2004. A
eficiência para os canais com o sinal de interesse, a taxa esperada de
seleção, o mecanismo de preparação de dados e os tempos de execução são
discutidos. Alguns casos da Física de interesse que será analisada pelo
experimento são detalhados.
\end{quotation}

\paragraph{Julho de 2005: \eng{Configuration of the ATLAS trigger}}
Apresentado no \eng{14th IEEE - NPSS Real Time Conference 2005 Nuclear Plasma
Sciences Society}. Referência \cite{aa:rt-05}.

\begin{quotation}
Este artigo discute um protótipo do sistema de configuração para os Altos
Níveis de Filtragem do experimento ATLAS (HLT). Tal sistema configurará os
algoritmos de filtragem no HLT para a deteção de eventos importantes para o
experimento. Apresenta-se uma visão geral do HLT e a implementação do
protótipo. Possíveis melhorias e tendências são discutidas no final do artigo.
\end{quotation}

\paragraph{Outubro de 2004: Os Filtros de Alto Nível do Experimento ATLAS}
Apresentado no XXV Encontro Nacional de Física de Partículas e
Campos. Referência \cite{aa:enfpc-04}.

\begin{quotation}
O Experimento ATLAS conta com um sistema de Filtragem bastante complexo e
dividido em 4 grandes sub-sistemas:
\begin{itemize}
\item O Primeiro Nível de Filtragem que realiza os primeiros passos da seleção
de eventos na cadeia de filtragem;
\item O \eng{Software Online}, responsável pelas áreas de controle e operação
do sistema; 
\item O Sistema de Fluxo de Dados (\eng{Dataflow}) que coordena a transmissão
e armazenamento dos dados do detetor do experimento;
\item Os Filtros de Alto Nível, que implementam os algoritmos de
discriminação, representando o topo da cadeia de seleção de eventos no ATLAS.
\end{itemize}
Para aumentar a portabilidade entre os algoritmos de filtragem desenvolvidos
por toda a comunidade do experimento, os desenvolvedores dos Filtros de Alto
Nível (ou simplesmente HLT; \eng{High-Level Triggers}) propuseram a
reutilização do ambiente de programação \eng{online} Athena dentro do sistema
que operará em tempo real. Para tal, o HLT utiliza as ferramentas propostas
pelo sub-sistema de Fluxo de Dados para coordenar as operações da
transferência de informação para dentro e para fora dos nós de processamento
sistema. Dentre outras restrições, o produto final deverá ser suficientemente
rápido e operável em tarefas concorrentes em máquinas (SMP) com vários
processadores rodando Linux.  Neste trabalho apresentamos alguns dos problemas
e soluções encontrados pelo grupo no desenvolvimento e teste do conjunto de
bibliotecas que compõe o HLT.
\end{quotation}

\paragraph{Setembro de 2004: \eng{Implementation and Performance of the
High Level Trigger Electron and Photon Selection for the ATLAS Experiment at
the LHC}} Apresentado na conferência \eng{Computing in High Energy Physics and
Nuclear Physics}. Referência~\cite{aa:chep-04-01}.

\begin{quotation}
Os Altos Níveis de Filtragem do experimento ATLAS (HLT) serão descritos em
\eng{software} e executados em PCs comercialmente disponíveis. Sua
implementação é baseada na arquitetura padrão, orientada à objetos, disponível
para análise \eng{offline} no experimento. Este artigo traz uma visão geral da
implementação atual da seleção de elétrons e fótons no HLT. Os níveis de
desempenho desta implementação são avaliados tendo por base simulações de
Monte Carlo, comparando-se a eficiência para o sinal de interesse, a taxa de
eventos esperada, tempos de pré-processamento dos dados e para a discriminação
em si. Alguns exemplos de física são discutidos, assim como a experiência de
utilização deste sistema na bancada de testes combinada com feixe em 2004.
\end{quotation}

\paragraph{Agosto de 2004: \eng{The Second Level Trigger of the ATLAS
Experiment at CERN's LHC}} Publicado na \eng{IEEE Transactions on Nuclear
Science}. Referência \cite{aa:tns-2004}.

\begin{quotation}
O Sistema de Filtragem do ATLAS reduz a taxa de eventos interessantes a ser
gravada para análise \eng{offline} em 3 níveis sucessivos de filtragem,
passando de 40~MHz iniciais a 100~kHz, 2~kHz e, finalmente, 200~Hz. Os Filtros
de Alto Nível e o Sistema de Aquisição de Dados são projetados para se
beneficiarem do mercado de computação e interconexão doméstica (PC's sob
Ethernet) para atingir este nível de desempenho. Neste artigo, discutimos o
Segundo Nível de Filtragem (LVL2) e apresentam-se resultados de medida de
desempenho.
\end{quotation}

\paragraph{Junho de 2004: \eng{Algorithms for the ATLAS high-level
trigger}} Publicado na \eng{IEEE Transactions on Nuclear
Science}. Referência~\cite{aa:tns-04-05}.

\begin{quotation}
Seguindo um rigoroso processo de projeto e análise em \eng{software}, uma
arquitetura baseada em objetos foi desenvolvida para deduzir o Segundo e
Terceiro Níveis de Filtragem do experimento ATLAS, no LHC. Os componentes
funcionais neste sistema, responsáveis por gerar as decisões, são algoritmos
operando dentro desta arquitetura de \eng{software}. Aspectos relevantes deste
sistema são revistos neste artigo, tal como exemplos de algoritmos específicos
e seu desempenho em cortes ``verticais'' de várias estratégias de seleção.
\end{quotation}

\paragraph{Junho de 2004: \eng{The base-line DataFlow system of the
ATLAS Trigger and DAQ}} Publicado na revista \eng{IEEE Transactions on Nuclear
Science}. Referência~\cite{aa:tns-2004-3}

\begin{quotation}
As linhas de base do projeto e implementação do Sistema de Fluxo de Dados do
ATLAS são descritas. Os componentes principais, suas interações,
bandas-passantes e taxas são discutidos assim como seu desempenho em um
protótipo com 10\% da dimensão final do experimento. Este sistema é uma
combinação de componentes com projeto dedicado e aplicações multi-tarefas
implementadas em C++ e rodando em PCs num ambiente baseado no Sistema
Operacional Linux. Os nós de processamento são conectados por uma rede
Gigabit-Ethernet padrão.
\end{quotation}

\paragraph{Junho de 2004: \eng{ATLAS TDAQ data collection software}} Publicado
na revista \eng{IEEE Transactions on Nuclear Science}. Referência
\cite{aa:tns-04-04}.

\begin{quotation}
O Sistema de Coleta de Dados (DC) é um componente do Sistema de Filtragem e
Aquisição de dados do experimento ATLAS, no LHC. Ele é responsável pela
movimentação de dados do Sistema de Leitura do Detetor (\eng{Readout System},
ROS) para o Segundo Nível de Filtragem e para o Filtro de Eventos. Esta
funcionalidade é distribuída em várias aplicações rodando sobre uma plataforma
Linux, interconectada por Gigabit-Ethernet. Para o projeto e implementação
destas aplicações, um enfoque comum foi adotado. Tal enfoque leva a uma
implentação comum do \eng{software} de DC, provendo uma infraestrutura de
serviços de base a todas as aplicações deste sub-sistema.
\end{quotation}

\paragraph{Outubro de 2003: \eng{Architecture of the ATLAS Online
Physics-Selection Software at the LHC}.} Apresentado no \eng{8\eiro
ICATPP}. Referência \cite{aa:como-2003}.

\begin{quotation}
A filtragem de eventos no experimento ATLAS é organizada em dois níveis
distintos: o Segundo Nível de Filtragem e e Filtro de Eventos. Um enfoque
unificado para selecionar eventos em ambos os níveis foi escolhido. Desta
forma, um conjunto de rotinas de base foi projetada para maximizar o
compartilhamento das interfaces e componentes \eng{offline}, ainda que
mantendo uma flexibilidade suficientemente grande para atender aos requisitos
operacionais do Sistema de Filtragem, notavelmente aqueles relacionados ao
desempenho e robustez. Este artigo descreve a arquitetura e o projeto do
sistema de seleção de eventos e mostra como esta implementação é compatível
com os desafios do experimento.
\end{quotation}

\paragraph{Outubro de 2003: \eng{Studies for a common Event Selection
Software: from LVL2 to Offline reconstruction}} Publicado na revista \eng{IEEE
Transactions on Nuclear Science}. Referência \cite{aa:tns-2004-2}.

\begin{quotation}
A função primária dos Filtros de Alto Nível do experimento ATLAS será
implementada em dois níveis independentes de deteção operando em componentes
de \eng{software}.  Enquanto este enfoque provê um conjunto de interfaces de
programação unificado ao sistema de seleção de eventos, também impõe
requisitos bastante estritos nos componentes \eng{offline} que devem ser
re-utilizados. A decisão do Segundo Nível de Filtragem do ATLAS deve,
tipicamente, ser atingida no espaço temporal de 10 milissegundos e conta com
paralelismo em múltiplas tarefas de processamento. De forma a atender a estas
restrições, foram desenvolvidos protótipos que incorporam elementos do Sistema
de Fluxo de Dados do ATLAS, dos Filtros de Alto Nível e do Sistema de
Processamento \eng{Offline}. De forma a implementar um ambiente homogêneo de
programação para os Filtros de Alto Nível, um controlador para o sistema de
execução do Segundo Nível de Filtragem foi desenvolvido. Tanto com fatias do
sistema de seleção baseado no canal elétron/fóton quanto fatias de seleção de
múons, demonstra-se que o desempenho necessário poderá ser atingido se os
componentes \eng{offline} são cuidadosamente projetados e otimizados para que
também sejam aplicáveis aos Altos Níveis de Filtragem do experimento.
\end{quotation}

\paragraph{Agosto de 2003: \eng{Neural Particle Discrimination for Triggering 
Interesting Physics Channels with Calorimetry Data}}. Publicado
na revista \eng{Nuclear Instruments And Methods In Physics Research A -
Accelerators, Spectrometers, Detectors And Associated Equipament}. Referência
\cite{aa:nima-2003}. 

\begin{quotation}
Experimentos em Física de Altas Energias usam sistemas de validação
\eng{online} (filtros) para distingüir a Física conhecida de eventos
interessantes que devam ser analisados. Este artigo introduz um esquema de
filtragem para processadores com alta taxa de entrada, baseado em redes
neurais artificiais. A técnica é aplicada ao problema de discriminação
elétron/jato, presente no Segundo Nível de Filtragem do experimento ATLAS,
sendo construído no CERN, Suíça. A solução proposta possui um melhor
desempenho que aquela proposta atualmente no CERN, tanto no caráter
discriminatório quanto em velocidade, tornando-se um algoritmo candidato para
implementação no experimento.
\end{quotation}

%\paragraph{Maio de 2003: \eng{The Event Format User Requirements
%Document}} Nota técnica publicada no CERN. O chave de referência é
%\textsf{ATL-DQ-EN-0005}. Referência \cite{aa:ef-urd}.

%\begin{quotation}
%Este documento contém os requirimentos, restrições e casos de uso para uma
%biblioteca de formatação de eventos, a ser usada potencialmente por vários
%sub-sistemas do Sistema de Filtragem e Aquisição de Dados do ATLAS.
%\end{quotation}

%\paragraph{Maio de 2003: \eng{The Event Format Analysis and Design}}. Nota
%técnica publicada no CERN. O chave de referência é
%\textsf{ATL-DQ-EN-0006}. Referência \cite{aa:ef-ooad}.

%\begin{quotation}
%Este documento descreve a análise e projeto de uma biblioteca de leitura e
%escrita de eventos para o experimento ATLAS.
%\end{quotation}

\paragraph{Março de 2003: \eng{ATLAS-TDAQ DataCollection Software}}. Publicado
na revista \eng{IEEE Transactions on Nuclear Sciences}. Referência
\cite{aa:rt-2003}.

\begin{quotation}
O Sistema Coleção de Dados é um sub-sistema do Sistema de Filtragem e
Aquisição de Dados do experimento ATLAS. Ele é responsável pela movimentação
dos dados de cada evento, a partir do Sistema de Leitura até o Segundo Nível
de Filtragem e ao Filtro de Eventos. Esta funcionalidade é distribuída por
vários componentes de \eng{software} operando sob Linux em PC's
interconectados por Gigabit Ethernet. Para o projeto e implementação destas
aplicações, um enfoque comum foi adotado. Este enfoque nos guiou ao projeto e
implementação de um conjunto de interfaces compartilhadas ao sistema de
Coleção de Dados, provendo uma suíte de serviços comuns.
\end{quotation}

\paragraph{Março de 2003: \eng{The DataFlow System of the ATLAS Trigger and
DAQ}}. Apresentado no congresso \eng{Computing for High-Energy Physics'
2003}. Referência \cite{aa:chep-2003-2}.

\begin{quotation}
Este artigo apresenta o projeto e implementação de um protótipo do Sistema de
Fluxo de Dados do experimento ATLAS. Sua decomposição funcional é descrita e
medidas de desempenho para cada componente individual são mostradas.
\end{quotation}

\paragraph{Março de 2003: \eng{Architecture of the ATLAS High Level
Trigger Event SelectionSoftware}}. Apresentado no congresso
\eng{Computing for High-Energy Physics' 2003}. Referência
\cite{aa:chep-2003-3}

\begin{quotation}
Os Altos Níveis de Filtragem do experimento ATLAS consistem de dois passos: o
Segundo Nível e um Filtro de Eventos. Ambos serão implementados em
\eng{software}, operando, em sua maior parte, sob componentes domésticos. O
modelo da seleção de eventos em ambos também é unificado, de forma que um
conjunto de interfaces comuns foi projetado para maximizar a coerência, ainda
que mantendo bons níveis de flexibilidade para garantir a interoperabilidade
aos dois níveis. Este enfoque é ainda estendido para permitir que o mesmo
conjunto de programas e bibliotecas opere \eng{offline}, facilitando o
desenvolvimento. Este artigo descreve a arquitetura e o projeto deste sistema.
\end{quotation}

\paragraph{Março de 2003: \eng{Experience with multi-threaded C++ applications
in the ATLAS DataFlow Software}} Apresentado na conferência
\eng{Computing for High-Energy Physics' 2003}. Referência \cite{aa:chep-2003}.

\begin{quotation}
O Sistema de Fluxo de Dados do experimento ATLAS é responsável pela recepção,
estocagem e subseqüente movimentação tanto parcial quanto completa dos dados
de um evento para os Altos Níveis de Filtragem. O projeto deste conjunto de
bibliotecas e programas é baseado em uma metodologia orientada à objetos e sua
implementação baseia-se no uso de Tarefas (\eng{threads}) POSIX e na
Biblioteca de Modelos-Padrão do C++ (\eng{Standard Template Library}). Este
artigo apresenta nossa experiência com Linux, Tarefas POSIX e a Biblioteca de
Modelos-Padrão do C++ no ambiente \eng{online} do Sistema de Fluxo de Dados do
ATLAS.
\end{quotation}

\paragraph{Julho de 2002: \eng{ATLAS High-Level Trigger, Data
Acquisition and Controls Technical Design Report}} Este documento descreve o
projeto e construção do Sistema de Filtragem do experimento ATLAS em si,
contendo todos os detalhes, medidas de desempenho e operabilidade. Foi
publicado no CERN. Referência \cite{hlt-tdr}.

\paragraph{Outubro de 2001: Redes Neurais Especialistas para a Separação
Elétron-Jato usando Calorímetros Multi-camadas e Multi-segmentados}
Apresentado no XX Encontro Nacional de Física de Partículas e
Campos. Referência \cite{aa:enfpc-01}.

\begin{quotation}
O experimento ATLAS estará operacional no ano de 2006. O objetivo principal
deste experimento é a deteção do bóson de Higgs, usando entre outros tipos de
detetores, calorímetros. Um dos canais de deteção mais importantes no
experimento é o de elétrons com alta energia transversa, representando de 30 a
40\% do total das assinaturas a serem analisadas pelo Sistema de
Filtragem. Jatos (de partículas) confundem-se comumente com elétrons pela
forma que interagem com os calorímetros. Neste trabalho, apresentamos um
sistema de discriminação elétron-jato baseado em redes neurais especialistas,
utilizando os dados dos calorímetros do ATLAS. Este sistema, depois de
treinado, compacta o espaço de variáveis de entrada (células dos calorímetros)
em um subespaço que mantém os aspectos necessários para uma deteção eficiente
de elétrons. Os resultados apresentados se mostram melhores que os resultados
obtidos usando-se técnicas desenvolvidas no CERN, com o mesmo objetivo.
\end{quotation}

\paragraph{Outubro de 2000: Mapeamento em An\'eis para uma Separa\c{c}\~ao
Neuronal El\'etron-Jato usando Calor\'imetros Multi-camadas e
Multi-segmentados} Apresentado no XIX Encontro Nacional de Física de
Partículas e Campos. Referência \cite{aa:enfpc-00}.

\begin{quotation}
Prop\~oe-se neste trabalho que a an\'alise conduzida no Segundo N\'{\i}vel de
Filtragem do Experimento ATLAS, no CERN, seja feita por meio de processamento
neural sobre a regi\~ao de interesse previamente destacada pelo Primeiro
N\'{\i}vel nos Calor\'{\i}metros. Por depender do posicionamento da RoI no
detetor, os números de camadas, granularides e profundidades das c\'elulas do
calor\'{\i}metro s\~ao desconhecidos at\'e o momento da chegada do evento ao
sistema de an\'alise. Ainda assim, estima-se que o n\'umero de c\'elulas para
an\'alise estar\'a em torno de 1000 por RoI.  As efici\^encias de
separa\c{c}\~ao obtidas, tempos de execu\c{c}\~ao e uma compara\c{c}\~ao com a
efici\^encia de outros m\'etodos empregados para a mesma atividade s\~ao
discutidas.
\end{quotation}

\paragraph{Outubro de 1999: Integrando Plataformas e Algoritmos para o Segundo
Nível de Trigger do Experimento ATLAS} Apresentado no XVIII Encontro 
Nacional de Física de Partículas e Campos. Referência \cite{aa:enfpc-99}.

\begin{quotation}
Este artigo contém um sumário do trabalho realizado no âmbito dos estudos de
portabilidade da infraestrutura do fluxo de dados do Sistema de Filtragem,
originalmente escritos em C e operando em Sistemas Operacionais comerciais
para uma implementação orientada a objetos baseada em C++ rodando sobre
Linux. Ele discute as vantagens deste enfoque, tanto em termos de
mantenibilidade quanto do custo final de projeto.
\end{quotation}

\paragraph{Outubro de 1998: Um Protótipo do Sistema de Validação de
Nível 2 para as Condições do LHC} Apresentado no XVII Encontro Nacional de
Física de Partículas e Campos. Referência \cite{aa:enfpc-98}.

\begin{quotation}
O experimento ATLAS pretende comprovar a existência do bóson de Higgs. Para
tal, um grande sistema de deteção e aquisição vem sendo projetado. O sistema
aquisição tem a função de separar em tempo real interações originárias do
decaimento de um Higgs de física ordinária.

A filtragem de eventos no sistema de aquisição é concebida em três níveis, de
complexidade crescente e velocidade decrescente. O segundo nível pretende
utilizar redes de computadores pessoais (PC's) interconectados por rápidos
sistemas de rede. A escolha de fabricantes, sistemas operacionais e algoritmos
de processamento ainda não foi feita, mas esforços em prol desta decisão vêm
sendo realizados. Neste trabalho desenvolve-se uma fração do filtro de segundo
nível utilizando-se de processamento paralelo, redes neurais artificiais e
DSP's.
\end{quotation}

\paragraph{Setembro de 1998: \eng{Neural Classifiers Implemented In A
Transputer Based Parallel Machine}} Apresentado no \eng{International Meeting
on Vector and Parallel Processing (VECPAR)}. Referência
\cite{aa:vecpar-1998}.

\begin{quotation}
O projeto de um protótipo para o Segundo Nível de Filtragem para as condições
do LHC é realizado através de uma implementação em uma máquina paralela. Este
computador contém 16 nós de processamento baseados em \eng{transputers} e usa
processadores digitais de sinais (\eng{DSP's}) como unidades de
co-processamento para otimizar o processamento de dados. Um ambiente de
desenvolvimento em C é usado para obter das aplicações seu máximo
desempenho. A implementação é baseada na informação provida pelos quatro
sistemas de deteção do experimento ATLAS em duas fases: extração de
características e decisão global. A extração de características para
calorímetros e decisão global são desempenhadas por redes neurais
artificiais. Os parâmetros de pré-processamento e da rede neural são mantidos
em memória, e a função de ativação da rede é implementada usando uma tabela de
procura. Dados simulados para o Segundo Nível de Filtragem são utilizados para
a avaliação do desempenho deste sistema.
\end{quotation}

\typeout{ *************** End of file published.tex *************** }
