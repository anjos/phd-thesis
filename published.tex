%% Hello emacs, this is -*- latex -*-
\typeout{ ====================================================================}
\typeout{ This is file published.tex, created at 13-Jun-2004 }
\typeout{ Maintained by Andre Rabello dos Anjos <Andre.dos.Anjos@cern.ch> }
\typeout{ ====================================================================}

\newcommand{\sectioneng}[1]{\foreignlanguage{english}{#1\/}}

\chapter{Publicações}
\label{chap:published}

Durante a execução deste estudo, um conjunto de trabalhos foi publicados na
forma de artigos e notas técnicas. Este apêndice traz uma descrição deste
material em ordem cronológica reversa.

\paragraph{Agosto de 2006: \sectioneng{Deployment of the ATLAS High-Level
Trigger}}, IEEE Transactions on Nuclear Science. Referência~\ref{aa:tns-06}

\begin{quotation}
The ATLAS combined test beam in the second half of 2004 saw the first
deployment of the ATLAS High-Level Trigger (HLT). The next steps are
deployment on the pre-series farms in the experimental area during 2005,
commissioning and cosmics tests with the full detector in 2006 and collisions
in 2007. This paper reviews the experience gained in the test beam, describes
the current status and discusses the further enhancements to be made. We
address issues related to the dataflow, integration of selection algorithms,
testing, software distribution, installation and improvements.
\end{quotation}

\paragraph{Maio de 2006: \sectioneng{A configuration system for the ATLAS
trigger}}, Journal of Instrumentation, Maio de
2006. Referência~\ref{aa:jinst-06}

\begin{quotation}
The ATLAS detector at CERN's Large Hadron Collider will be exposed to
proton-proton collisions from beams crossing at 40 MHz that have to be
reduced to the few hundreds of Hz allowed by the storage systems. A
three-level trigger system has been designed to achieve this goal. We describe
the configuration system under construction for the ATLAS trigger chain. It
provides the trigger system with all the parameters required for decision
taking and to record its history. The same system configures the event
reconstruction, Monte Carlo simulation and data analysis, and provides tools
for accessing and manipulating the configuration data in all contexts.
\end{quotation}

\paragraph{Abril de 2006: \sectioneng{Neural triggering system operating on
high resolution calorimetry information}} na Nuclear Instruments and Methods
in Physics Research. Referência~\cite{aa:nim-06}

\begin{quotation}
This paper presents an electron/jet discriminator system for operating at the
Second Level Trigger of ATLAS. The system processes calorimetry data and
organizes the regions of interest in the calorimeter in the form of concentric
ring sums of energy deposition, so that both signal compaction and high
performance can be achieved. The ring information is fed into a feed forward
neural discriminator. This implementation resulted on a 97\% electron
detection efficiency for a false alarm of 3\%. The full discrimination chain
could still be executed in less than 500 us.
\end{quotation}

\paragraph{Fevereiro de 2006: \setioneng{Testing on a Large Scale : running
the ATLAS Data Acquisition and High Level Trigger Software on 700 PC Nodes}}
Computing In High Energy and Nuclear Physics  CHEP 2006 , Mumbai, India , 13 -
17 Feb 2006. Referência~\cite{aa:chep-06-01}

\begin{quotation}
The ATLAS Data Acquisition (DAQ) and High Level Trigger (HLT) software system
will be comprised initially of 2000 PC nodes which take part in the control,
event readout, second level trigger and event filter operations. This high
number of PCs will only be purchased before data taking in 2007. The large
CERN IT LXBATCH facility provided the opportunity to run in July 2005 online
functionality tests over a period of 5 weeks on a stepwise increasing farm
size from 100 up to 700 PC dual nodes. The interplay between the control and
monitoring software with the event readout, event building and the trigger
software has been exercised the first time as an integrated system on this
large scale. New was also to run algorithms in the online environment for the
trigger selection and in the event filter processing tasks on a larger
scale. A mechanism has been developed to package the offline software together
with the DAQ/HLT software and to distribute it via peer-to-peer software
efficiently to this large pc cluster. The findings obtained during the tests
lead to many immediate improvements in the software. Trend analysis allowed
identifying critical areas. Running an online system on a cluster of 700 nodes
successfully was found to be especially sensitive to the reliability of the
farm as well as the DAQ/HLT system itself and the future development will
concentrate on fault tolerance and stability.
\end{quotation}

\paragraph{Julho de 2005: \sectioneng{Overview of the High-Level Trigger
Electron and Photon Selection for the ATLAS Experiment at the LHC}} na 14th
IEEE - NPSS Real Time Conference 2005 Nuclear Plasma Sciences
Society. Referência~\cite{aa:rt-05-01}

\begin{quotation}
The ATLAS experiment at the Large Hadron Collider (LHC) will face the
challenge of efficiently selecting interesting candidate events in pp
collisions at 14 TeV center-of-mass energy, whilst rejecting the enormous
number of background events. The High-Level Trigger (HLT = second level
trigger and Event Filter), which is a software based trigger will need to
reduce the level-1 output rate of ~75 kHz to ~200 Hz written out to mass
storage. In this talk an overview of the current physics and system
performance of the HLT selection for electrons and photons is given. The
performance has been evaluated using Monte Carlo simulations and has been
partly demonstrated in the ATLAS testbeam in 2004. The efficiency for the
signal channels, the rate expected for the selection, the global data
preparation and execution times will be highlighted. Furthermore, some physics
examples will be discussed to demonstrate that the triggers are well adapted
for the physics programme envisaged at the LHC.
\end{quotation}

\paragraph{Julho de 2005: \sectioneng{Configuration of the ATLAS trigger}} na
14th IEEE - NPSS Real Time Conference 2005 Nuclear Plasma Sciences
Society. Referência~\cite{aa:rt-05}

\begin{quotation}
The ATLAS detector at CERN's LHC will be exposed to proton-proton collisions
at a rate of 40 MHz. In order to reduce the data rate to about 200 Hz, only
potentially interesting events are selected by a three-level trigger
system. Its first level is implemented in electronics and firmware whereas the
higher trigger levels are based on software. To prepare the full trigger chain
for the online event selection according to a certain strategy, a system is
being set up that provides the relevant configuration information -
e.g. values for hardware registers in level-1 or parameters of high-level
trigger algorithms - and stores the corresponding history. The same
information is used to configure the offline trigger simulation. In this
presentation an overview of the ATLAS trigger system is given concentrating on
the event selection strategy and its description. The technical implementation
of the configuration system is summarized.
\end{quotation}

\paragraph{Setembro de 2004: \sectioneng{Implementation and Performance of the
High Level Trigger Electron and Photon Selection for the ATLAS Experiment at
the LHC}}, na conferência Computing in High Energy Physics and Nuclear
Physics, em Interlaken, na Suíça. Referência~\cite{aa:chep-04-01}

\begin{quotation}
The ATLAS experiment at the Large Hadron Collider (LHC) will face the
challenge of efficiently selecting interesting candidate events in pp
collisions at 14 TeV center of mass energy, while rejecting the enormous
number of background events, stemming from an interaction rate of up to 10^9
Hz. The First Level trigger will reduce this rate to around O(100
kHz). Subsequently, the High Level Trigger (HLT), which is comprised of the
Second Level trigger and the Event Filter, will need to further reduce this
rate by a factor of O(10^3). The HLT selection is software based and will be
implemented on commercial CPUs, using a common framework built on the standard
ATLAS object oriented software architecture. In this paper an overview of the
current implementation of the selection for electrons and photons in the HLT
is given. The performance of this implementation has been evaluated using
Monte Carlo simulations in terms of the efficiency for the signal channels,
rate expected for the selection, data preparation times, and algorithm
execution times. Besides the efficiency and rate estimates, some physics
examples will be discussed, showing that the triggers are well adapted for the
physics programme envisaged at LHC. The electron and photon trigger software
is also being exercised at the ATLAS 2004 Combined Test Beam, where components
from all ATLAS subdetectors are taking data together along the H8 SPS
extraction line; from these tests a validation of the selection architecture
chosen in a real on-line environment is expected.
\end{quotation}

\paragraph{Junho de 2004: \sectioneng{Algorithms for the ATLAS high-level
trigger}}. Referência~\cite{aa:tns-04-05}. 

\begin{quotation}
Following rigorous software design and analysis methods, an object-based
architecture has been developed to derive the second- and third-level trigger
decisions for the future ATLAS detector at the LHC. The functional components
within this system responsible for generating elements of the trigger
decisions are algorithms running within the software architecture. Relevant
aspects of the architecture are reviewed along with concrete examples of
specific algorithms and their performance in "vertical" slices of various
physics selection strategies.
\end{quotation}

\paragraph{Junho de 2004: \sectioneng{The base-line DataFlow system of the ATLAS
Trigger and DAQ}} Na revista \sectioneng{IEEE Transactions on Nuclear
Science}, volume 51, número 3. Referência \cite{aa:tns-2004-3}.

\begin{quotation}
\sectioneng{The base-line design and implementation of the ATLAS DAQ DataFlow
system is described. The main components realizing the DataFlow system, their
interactions, bandwidths and rates are being discussed and performance
measurements on a 10\% scale prototype for the final Atlas TDAQ DataFlow system
are presented. \\
This prototype is a combination of custom design components and of
multi-threaded software applications implemented in C++ and running in a Linux
environment on commercially available PCs interconnected by a fully switched
gigabit Ethernet network.}
\end{quotation}

\paragraph{Junho de 2004: \sectioneng{ATLAS TDAQ data collection software} na
\sectioneng{IEEE Transactions on Nuclear Science}, volume 51, número
3. \cite{aa:tns-2004-3}. 

\begin{quotation}
The DataCollection (DC) is a subsystem of the ATLAS Trigger and DAQ system. It
is responsible for the movement of event data from the ReadOut subsystem to
the Second Level Trigger and to the Event Filter. This functionality is
distributed on several software applications running on Linux PCs
interconnected with Gigabit Ethernet. For the design and implementation of
these applications a common approach has been adopted. This approach leads to
the design and implementation of a common DC software framework providing a
suite of common services. 10 Refs.
\end{quotation}

\paragraph{Outubro de 2003: \sectioneng{The Second Level Trigger of the ATLAS
Experiment at CERN's LHC}} Este trabalho foi apresentado no \sectioneng{Nucler
Science Symposium}, sediado em \sectioneng{Portland}, \sectioneng{Oregon},
EUA. Este trabalho também será publicado na revista \sectioneng{IEEE
Transactions on Nuclear Science}, volume 51, número 3. A revista será
publicada em junho de 2004 e, portanto, o número das páginas ainda não está
disponível. Referência
\cite{aa:tns-2004}.

\begin{quotation}
\sectioneng{The ATLAS trigger reduces the rate of interesting events to be
recorded for off-line analysis in 3 successive levels from 40 MHz to ~100 kHz,
~2 kHz and ~200 Hz. The High Level Triggers and Data Acquisition System are
designed to profit from commodity computing and networking components to
achieve the required performance. In this paper, we discuss Data Flow aspects
of the design of the Second Level Trigger (LVL2) and present results of
performance measurements.}
\end{quotation}

\paragraph{Outubro de 2003: \sectioneng{Architecture of the ATLAS Online
Physics-Selection Software at the LHC}.} Este artigo foi apresentado no
\sectioneng{8th ICATPP Conference Proceedings} em Villa Erba, Como, Itália. 
Referência \cite{aa:como-2003}.

\begin{quotation}
\sectioneng{Given the extremely high bunch crossing rate foreseen at the Large
Hadron Collider (LHC) and the general-purpose nature of the ATLAS particle
physics experiment, after the hardware-based first level trigger, an efficient
and flexible softwa re is needed for the online selection of physics
events. This filtering of event s is organized in two levels: the second level
trigger and the event filter. A c oherent approach to event selection across
both levels has been taken. Thus a co mmon core software framework has been
designed to maximise the usage of offline interfaces and software components,
whilst allowing sufficient flexibility to me et the different interfaces and
requirements of the two different levels, notabl y those of performance and
robustness. This paper describes the architecture and high level design of the
selection software and shows how the implementation me ets the challenges of
the ATLAS environment.}
\end{quotation}

\paragraph{Outubro de 2003: \sectioneng{Studies for a common Event Selection
Software: from LVL2 to Offline reconstruction}} Este trabalho foi apresentado
no \sectioneng{Nucler Science Symposium}, sediado em \sectioneng{Portland},
\sectioneng{Oregon}, EUA. Este trabalho também será publicado na revista
\sectioneng{IEEE Transactions on Nuclear Science}, volume 51, número 3. A
revista será publicada em junho de 2004 e, portanto, o número das páginas
ainda não está disponível. Referência \cite{aa:tns-2004-2}.

\begin{quotation}
\sectioneng{The ATLAS High Level Trigger's primary function of event selection
will be accomplished with a Level-2 trigger farm and an Event Filter farm,
both running software components developed in the ATLAS offline reconstruction
framework. While this approach provides a unified software framework for event
selection, it poses strict requirements on offline components critical for the
Level-2 trigger. A Level-2 decision in ATLAS must typically be accomplished
within 10 ms and with multiple event processing in concurrent threads. In
order to address these constraints, prototypes have been developed that
incorporate elements of the ATLAS Dataflow, High Level Trigger, and
offline framework software. To realize a homogeneous software environment for
offline components in the High Level Trigger, the Level-2 Steering Controller
was developed. With electron/gamma and muon-selection slices it has been shown
that the required performance can be reached, if the offline components used
are carefully designed and optimized for the application in the High Level
Trigger.}
\end{quotation}

\paragraph{Agosto de 2003: \sectioneng{Neural Particle Discrimination for Triggering 
Interesting Physics Channels with Calorimetry Data}} Este artigo foi publicado
na revista \sectioneng{Nuclear Instruments And Methods In Physics Research A -
Accelerators, Spectrometers, Detectors And Associated Equipament}, volume 502,
número 2, páginas 713 a 715. Referência \cite{aa:nima-2003}.

\begin{quotation}
\sectioneng{High Energy Physics experiments use on-line event validators
(triggers) to distinguish known physics from unstudied physics events. This
article introduces a triggering scheme for high input rate processors, based
on neural networks. The technique is applied to the Electron/Jet
discrimination problem, present at the Second Level Trigger of the ATLAS
experiment, being constructed at CERN, Switzerland. The proposed solution
outperforms the scheme adopted nowadays at CERN, both in discrimination
efficiency and performance, becoming a candidate algorithm for implementation
at the experiment.}
\end{quotation}

\paragraph{Maio de 2003: \sectioneng{The Event Format User Requirements
Document}} Nota técnica publicada no CERN. O chave de referência é
\textsf{ATL-DQ-EN-0005}. O documento pode ser permanentemente acessado como
especificado na referência \cite{aa:ef-urd}.

\begin{quotation}
\sectioneng{This document includes the requirements, constraints and use-cases
for a common event format library, to be used potentially by several
Trigger/DAQ subsystems.}
\end{quotation}

\paragraph{Maio de 2003: \sectioneng{The Event Format Analysis and Design}} Nota
técnica publicada no CERN. O chave de referência é
\textsf{ATL-DQ-EN-0006}. O documento pode ser permanentemente acessado como
especificado na referência \cite{aa:ef-ooad}.

\begin{quotation}
\sectioneng{This document describes the Analysis and Design of a multi-purpose
library for the ATLAS experiment that can read and write event formatted
data. Library users are guaranteed to be able to exchange data produced by
ATLAS detectors, triggers and their equivalent simulation environments.}
\end{quotation}

\paragraph{Março de 2003: \sectioneng{ATLAS-TDAQ DataCollection Software}} Este
trabalho foi apresentado no congresso \sectioneng{Real-Time' 2003} e publicado nos
anais do mesmo. Este trabalho também será publicado na revista
\sectioneng{IEEE Transactions on Nuclear Sciences}, volume 51, número 3. A revista
será publicada em junho de 2004 e, portanto, o número das páginas ainda não
está disponível. Referência \cite{aa:rt-2003}.

\begin{quotation}
\sectioneng{The DataCollection is a subsystem of the ATLAS Trigger and DAQ
system. It is responsible for the movement of event data from the ReadOut
subsystem to the Second Level Trigger and to the Event Filter. This
functionality is distributed on several software applications running on Linux
PCs interconnected with Gigabit Ethernet. For the design and implementation of
these applications a common approach has been adopted. This approach leads to
the design and implementation of a common DataCollection software framework
providing a suite of common services.}
\end{quotation}

\paragraph{Março de 2003: \sectioneng{The DataFlow System of the ATLAS Trigger and
DAQ}} Este trabalho foi apresentado no congresso \sectioneng{Computing for
High-Energy Physics' 2003}, em \sectioneng{La Jolla}, Califórnia, EUA. Referência
\cite{aa:chep-2003-2}.

\begin{quotation}
\sectioneng{This paper presents the design and prototype implementation of the
DataFlow system of the ATLAS experiment. Its functional decomposition is
described and performance measurements for each individual component are
shown.}
\end{quotation}

\paragraph{Março de 2003: \sectioneng{Architecture of the ATLAS High Level
Trigger Event SelectionSoftware}} Este trabalho foi apresentado no congresso
\sectioneng{Computing for High-Energy Physics' 2003}, em \sectioneng{La
Jolla}, Califórnia, EUA. Referência \cite{aa:chep-2003-3}

\begin{quotation}
\sectioneng{The ATLAS High Level Trigger (HLT) consists of two selection
steps: the second l evel trigger and the event filter. Both will be
implemented in software, running on mostly commodity hardware. Both levels
have a coherent approach to event sel ection, so a common core software
framework has been designed to maximise this coherency, while allowing
sufficient flexibility to meet the different interfaces and requirements of
the two different levels. The approach is extended further to allow the
software to run in an off-line simulation and reconstruction enviro nment for
the purposes of development. This paper will describe the architecture and
high level design of the software.}
\end{quotation}

\paragraph{Março de 2003: \sectioneng{Experience with multi-threaded C++ applications
in the ATLAS DataFlow Software}} Publicado nos anais do congresso
\sectioneng{Computing for High-Energy Physics' 2003}, após apresentação do trabalho
no mesmo congresso. O local de apresentação foi \sectioneng{La Jolla}, Califórnia,
EUA. Referência \cite{aa:chep-2003}.

\begin{quotation}
\sectioneng{The DataFlow is sub-system of the ATLAS data acquisition responsible for the
reception, buffering and subsequent movement of partial and full event data to
the higher level triggers: Level 2 and Event Filter. The design of the
software is based on OO methodology and its implementation relies heavily on
the use of posix threads and the Standard Template Library. This article
presents our experience with Linux, posix threads and the Standard Template
Library in the real time environment of the ATLAS data flow.}
\end{quotation}

\paragraph{Julho de 2002: \sectioneng{ATLAS High-Level Trigger, Data Acquisition and
Controls Technical Design Report}} Este documento descreve o projeto e
construção do Sistema de Filtragem do experimento ATLAS. Foi publicado no
CERN. Referência \cite{hlt-tdr}.

\paragraph{Agosto de 1998: \sectioneng{A neural online triggering system based
on parallel processing}} Publicado na IEEE Transactions on Nuclear Science. 

\begin{quotation}
The study of a prototype of the second-level triggering system for operation
at LHC conditions is addressed by means of a parallel machine
implementation. The 16 node transputer based machine uses a fast digital
signal processor acting as a coprocessor for optimizing signal processing
applications. A C-language development environment is used for running all
applications at ultimate speed. The implementation is based on information
supplied by four detectors and includes two phases of system operation:
feature extraction and global decision. Feature extraction for calorimeters
and global decision processing are performed by means of neural
networks. Preprocessing and neural network parameters rest in memory and the
activation function is implemented using a look up table. Simulated data for
the second-level trigger operation are used for performance evaluation.
\end{quotation}

\typeout{ *************** End of file published.tex *************** }
