%% Hello emacs, this is -*- latex -*-
\typeout{ ====================================================================}
\typeout{ This is file published.tex, created at 13-Jun-2004 }
\typeout{ Maintained by Andre Rabello dos Anjos <Andre.dos.Anjos@cern.ch> }
\typeout{ ====================================================================}

\newcommand{\sectioneng}[1]{\foreignlanguage{english}{#1\/}}

\chapter{Publicações e Projetos}
\label{chap:published}

Durante a execução deste estudo, alguns trabalhos foram publicados na forma de
artigos e notas técnicas. Ademais, a participação da UFRJ, através da estadia
do doutorando no CERN, foi concretizada através de projetos realizados em
parceria direta. Neste capítulo, descreve-se cada um dos itens relevantes a
este trabalho. O período coberto inicia-se em abril de 2001 e, estende-se até
o presente momento. As citações estão organizadas pela data de publicação e,
no caso de projetos, pela data de início.

\section{Trabalhos Publicados}

\paragraph{Julho de 2002: \sectioneng{ATLAS High-Level Trigger, Data Acquisition and
Controls Technical Design Report}} Este documento descreve o projeto e
construção do Sistema de Filtragem do experimento ATLAS. Foi publicado no
CERN. Referência \cite{hlt-tdr}.

\paragraph{Março de 2003: \sectioneng{Experience with multi-threaded C++ applications
in the ATLAS DataFlow Software}} Publicado nos anais do congresso
\sectioneng{Computing for High-Energy Physics' 2003}, após apresentação do trabalho
no mesmo congresso. O local de apresentação foi \sectioneng{La Jolla}, Califórnia,
EUA. Referência \cite{aa:chep-2003}.

\begin{quotation}
\sectioneng{The DataFlow is sub-system of the ATLAS data acquisition responsible for the
reception, buffering and subsequent movement of partial and full event data to
the higher level triggers: Level 2 and Event Filter. The design of the
software is based on OO methodology and its implementation relies heavily on
the use of posix threads and the Standard Template Library. This article
presents our experience with Linux, posix threads and the Standard Template
Library in the real time environment of the ATLAS data flow.}
\end{quotation}

\paragraph{Março de 2003: \sectioneng{ATLAS-TDAQ DataCollection Software}} Este
trabalho foi apresentado no congresso \sectioneng{Real-Time' 2003} e publicado nos
anais do mesmo. Este trabalho também será publicado na revista
\sectioneng{IEEE Transactions on Nuclear Sciences}, volume 51, número 3. A revista
será publicada em junho de 2004 e, portanto, o número das páginas ainda não
está disponível. Referência \cite{aa:rt-2003}.

\begin{quotation}
\sectioneng{The DataCollection is a subsystem of the ATLAS Trigger and DAQ
system. It is responsible for the movement of event data from the ReadOut
subsystem to the Second Level Trigger and to the Event Filter. This
functionality is distributed on several software applications running on Linux
PCs interconnected with Gigabit Ethernet. For the design and implementation of
these applications a common approach has been adopted. This approach leads to
the design and implementation of a common DataCollection software framework
providing a suite of common services.}
\end{quotation}

\paragraph{Março de 2003: \sectioneng{The DataFlow System of the ATLAS Trigger and
DAQ}} Este trabalho foi apresentado no congresso \sectioneng{Computing for
High-Energy Physics' 2003}, em \sectioneng{La Jolla}, Califórnia, EUA. Referência
\cite{aa:chep-2003-2}.

\begin{quotation}
\sectioneng{This paper presents the design and prototype implementation of the
DataFlow system of the ATLAS experiment. Its functional decomposition is
described and performance measurements for each individual component are
shown.}
\end{quotation}

\paragraph{Architecture of the ATLAS High Level Trigger Event Selecti
on Software} Este trabalho foi apresentado no congresso \sectioneng{Computing
for High-Energy Physics' 2003}, em \sectioneng{La Jolla}, Califórnia,
EUA. Referência \cite{aa:chep-2003-3}

\begin{quotation}
\sectioneng{The ATLAS High Level Trigger (HLT) consists of two selection
steps: the second l evel trigger and the event filter. Both will be
implemented in software, running on mostly commodity hardware. Both levels
have a coherent approach to event sel ection, so a common core software
framework has been designed to maximise this coherency, while allowing
sufficient flexibility to meet the different interfaces and requirements of
the two different levels. The approach is extended further to allow the
software to run in an off-line simulation and reconstruction enviro nment for
the purposes of development. This paper will describe the architecture and
high level design of the software.}
\end{quotation}

\paragraph{Maio de 2003: \sectioneng{The Event Format User Requirements
Document}} Nota técnica publicada no CERN. O chave de referência é
\textsf{ATL-DQ-EN-0005}. O documento pode ser permanentemente acessado como
especificado na referência \cite{aa:ef-urd}.

\begin{quotation}
\sectioneng{This document includes the requirements, constraints and use-cases
for a common event format library, to be used potentially by several
Trigger/DAQ subsystems.}
\end{quotation}

\paragraph{Maio de 2003: \sectioneng{The Event Format Analysis and Design}} Nota
técnica publicada no CERN. O chave de referência é
\textsf{ATL-DQ-EN-0006}. O documento pode ser permanentemente acessado como
especificado na referência \cite{aa:ef-ooad}.

\begin{quotation}
\sectioneng{This document describes the Analysis and Design of a multi-purpose
library for the ATLAS experiment that can read and write event formatted
data. Library users are guaranteed to be able to exchange data produced by
ATLAS detectors, triggers and their equivalent simulation environments.}
\end{quotation}

\paragraph{Agosto de 2003: \sectioneng{Neural Particle Discrimination for Triggering 
Interesting Physics Channels with Calorimetry Data}} Este artigo foi publicado
na revista \sectioneng{Nuclear Instruments And Methods In Physics Research A -
Accelerators, Spectrometers, Detectors And Associated Equipament}, volume 502,
número 2, páginas 713 a 715. Referência \cite{aa:nima-2003}.

\begin{quotation}
\sectioneng{High Energy Physics experiments use on-line event validators
(triggers) to distinguish known physics from unstudied physics events. This
article introduces a triggering scheme for high input rate processors, based
on neural networks. The technique is applied to the Electron/Jet
discrimination problem, present at the Second Level Trigger of the ATLAS
experiment, being constructed at CERN, Switzerland. The proposed solution
outperforms the scheme adopted nowadays at CERN, both in discrimination
efficiency and performance, becoming a candidate algorithm for implementation
at the experiment.}
\end{quotation}

\paragraph{Outubro de 2003: \sectioneng{Architecture of the ATLAS Online
Physics-Selection Software at LHC}.} Este artigo foi apresentado no
\sectioneng{8th ICATPP Conference Proceedings} em Villa Erba, Como, Itália. 
Referência \cite{aa:como-2003}.

\begin{quotation}
\sectioneng{Given the extremely high bunch crossing rate foreseen at the Large
Hadron Collider (LHC) and the general-purpose nature of the ATLAS particle
physics experiment, after the hardware-based first level trigger, an efficient
and flexible softwa re is needed for the online selection of physics
events. This filtering of event s is organized in two levels: the second level
trigger and the event filter. A c oherent approach to event selection across
both levels has been taken. Thus a co mmon core software framework has been
designed to maximise the usage of offline interfaces and software components,
whilst allowing sufficient flexibility to me et the different interfaces and
requirements of the two different levels, notabl y those of performance and
robustness. This paper describes the architecture and high level design of the
selection software and shows how the implementation me ets the challenges of
the ATLAS environment.}
\end{quotation}

\paragraph{Outubro de 2003: \sectioneng{Studies for a common Event Selection
Software: from LVL2 to Offline reconstruction}} Este trabalho foi apresentado
no \sectioneng{Nucler Science Symposium}, sediado em \sectioneng{Portland},
\sectioneng{Oregon}, EUA. Este trabalho também será publicado na revista
\sectioneng{IEEE Transactions on Nuclear Science}, volume 51, número 3. A
revista será publicada em junho de 2004 e, portanto, o número das páginas
ainda não está disponível. Referência \cite{aa:tns-2004-2}.

\begin{quotation}
\sectioneng{The ATLAS High Level Trigger's primary function of event selection
will be accomplished with a Level-2 trigger farm and an Event Filter farm,
both running software components developed in the ATLAS offline reconstruction
framework. While this approach provides a unified software framework for event
selection, it poses strict requirements on offline components critical for the
Level-2 trigger. A Level-2 decision in ATLAS must typically be accomplished
within 10 ms and with multiple event processing in concurrent threads. In
order to address these constraints, prototypes have been developed that
incorporate elements of the ATLAS Dataflow, High Level Trigger, and
offline framework software. To realize a homogeneous software environment for
offline components in the High Level Trigger, the Level-2 Steering Controller
was developed. With electron/gamma and muon-selection slices it has been shown
that the required performance can be reached, if the offline components used
are carefully designed and optimized for the application in the High Level
Trigger.}
\end{quotation}

\paragraph{Outubro de 2003: \sectioneng{The Second Level Trigger of the ATLAS
Experiment at CERN's LHC}} Este trabalho foi apresentado no \sectioneng{Nucler
Science Symposium}, sediado em \sectioneng{Portland}, \sectioneng{Oregon},
EUA. Este trabalho também será publicado na revista \sectioneng{IEEE
Transactions on Nuclear Science}, volume 51, número 3. A revista será
publicada em junho de 2004 e, portanto, o número das páginas ainda não está
disponível. Referência
\cite{aa:tns-2004}.

\begin{quotation}
\sectioneng{The ATLAS trigger reduces the rate of interesting events to be
recorded for off-line analysis in 3 successive levels from 40 MHz to ~100 kHz,
~2 kHz and ~200 Hz. The High Level Triggers and Data Acquisition System are
designed to profit from commodity computing and networking components to
achieve the required performance. In this paper, we discuss Data Flow aspects
of the design of the Second Level Trigger (LVL2) and present results of
performance measurements.}
\end{quotation}

\paragraph{Junho de 2004: \sectioneng{The base-line DataFlow system of the ATLAS
Trigger and DAQ}} A ser publicado na revista \sectioneng{IEEE Transactions on
Nuclear Science}, volume 51, número 3. A revista será publicada em junho de
2004 e, portanto, o número das páginas ainda não está disponível. Referência
\cite{aa:tns-2004-3}.

\begin{quotation}
\sectioneng{The base-line design and implementation of the ATLAS DAQ DataFlow
system is described. The main components realizing the DataFlow system, their
interactions, bandwidths and rates are being discussed and performance
measurements on a 10\% scale prototype for the final Atlas TDAQ DataFlow system
are presented. \\
This prototype is a combination of custom design components and of
multi-threaded software applications implemented in C++ and running in a Linux
environment on commercially available PCs interconnected by a fully switched
gigabit Ethernet network.}
\end{quotation}

\section{Projetos}

Durante a estadia no CERN, embora a partipação tenha se dado no escopo do
projeto do sistema de filtragem, alguns tópicos específicos tiveram especial
atenção. Nesta seção destacam-se as principais contribuições durante o período
do doutorando no exterior.

Os projetos continuam ativos embora, em alguns casos, tenham atingido
maturidade e requeiram somente esforço de manutenção e correção de
problemas. Outros ainda estão em fase de desenvolvimento. Distinguem-se, no
decorrer do texto, projetos que chegaram à fase de estabilidade operacional
para o Sistema de Filtragem, daqueles que ainda estão sendo produzidos.

\paragraph{A Unidade de Processamento do LVL2}

Este projeto foi desenvolvido dentro da comunidade de Fluxo de Dados. Como
explanado anteriormente, a Unidade de Processamento do LVL2 ou L2PU é parte
integrante do sistema de aquisição e hospeda os algoritmos de filtragem do
ambiente Athena, enquanto executados \eng{online}.

Este projeto teve início em janeiro de 2002 e foi considerado maduro em meados
de 2003. Não há mais desenvolvimentos nesta área. O projeto é apenas mantido e
incrementado para a correção de erros e adição de pequenos ajustes.

\paragraph{Pseudo-ROS}

Este projeto foi desenvolvido dentro da comunidade de Fluxo de Dados. O
Pseudo-ROS recebe os dados dos eventos processados por uma L2PU. Grande parte
do código deste processador foi desenvolvida em parceria com a UFRJ.

Este projeto teve início em meados de 2002 e foi considerado maduro em meados
de 2003. Não há mais desenvolvimentos nesta área. O projeto é apenas mantido e
incrementado para a correção de erros e adição de pequenos ajustes.

\paragraph{Biblioteca de Formatação de Eventos}

Este projeto foi desenvolvido dentro da comunidade de Fluxo de Dados. Esta
biblioteca pode ler e escrever eventos gerados pelo detetor ATLAS e é usada
por toda a comunidade do experimento (\eng{online} e \eng{offline}) em
múltiplas aplicações. Este produto foi totalmente desenvolvido pela
Universidade e encontra-se em fase de manutenção apenas. Ele foi considerado
maduro em meados de 2003.

\paragraph{Relatório de Erros}

Este projeto está sendo desenvolvido dentro da comunidade de Fluxo de
Dados. Esta biblioteca proverá, quando acabada, uma interface amigável para
que as aplicações do Fluxo de Dados (L2PU, L2SV, PROS, ROS, etc) possam
relatar erros ou avisos decorrentes de situações inusitadas. Este sistema
estará intimamente ligado ao sistema de controle do experimento, que contará
ultimamente com um programa controlador, suficientemente inteligente, que
aplicará ações corretivas ao Sistema de Filtragem, com intervenção mínima do
operador do \eng{run}.

A biblioteca de relatório de erros começou a ser desenvolvida em novembro de
2003 e tem data prevista de término para outubro de 2004.

\paragraph{Biblioteca de Pré-processamento de Dados de Calorímetros}

Este projeto implementou uma biblioteca rápida de pré-processamento de dados
para os Calorímetros do ATLAS. Este projeto teve o único objetivo de provar
que era possível vencer os enormes tempos de processamento requeridos pelo
ambiente Athena. Os resultados tiveram um impacto bastante positivo na
comunidade \eng{Offline}, tendo a maior parte do código sido transportada para
aquele conjunto de bibliotecas. Este projeto conseguiu reduzir, em quase 50
vezes, o tempo de processamento para objetos calorimétricos. Embora continue
presente no Sistema de Filtragem, o projeto em si está inativo.

\paragraph{Athena Multi-threaded}

Este projeto foi encaminhado com a participação da Universidade. O projeto
consiste do transporte das rotinas do ambiente \eng{Offline}, para que o
Athena possa ser executado dentro do Sistema de Filtragem. Este projeto rendeu
muitos frutos e, também, uma publicação em revista (\eng{Studies for a common
Event Selection Software: from LVL2 to Offline reconstruction}) especializada,
como mencionado na seção anterior.

Este projeto foi iniciado em meados de 2002 e continua ativo. Não é possível
prever o tempo que demorará até que amadureça e possa ser usado em escala de
produção.

\typeout{ *************** End of file published.tex *************** }
