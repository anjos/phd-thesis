%% Hello emacs, this is -*- latex -*-
\typeout{ ====================================================================}
\typeout{ This is file prefacio.tex, created at 13-Jun-2004 }
\typeout{ Maintained by Andre Rabello dos Anjos <Andre.dos.Anjos@cern.ch> }
\typeout{ ====================================================================}

\chapter{Introdução}
%%\addcontentsline{toc}{chapter}{\numberline{}Pref\'acio}

Sistemas eletrônicos de aquisição de dados são comumente empregados em muitos
campos da engenharia. Exemplos podem ser encontrados na captura de áudio,
vídeo, sinais de satélite, rádio, biológicos e cartográficos dentre outros.

Atualmente, em vários dos problemas, o sistema de deteção de sinais é bastante
complexo, impingindo complexidades também ao sistema de aquisição de dados. O
sistema de deteção captando o sinal de interesse pode estar distribuído em
múltiplas localidades ou requisitar uma velocidade de processamento que torne
impraticável a utilização direta dos produtos disponíveis no mercado. Nestes
casos, é comum também distribuir o sistema de aquisição, combinando-o
posteriormente através módulos centralizados.

Dependendo do domínio do problema, o sinal adquirido deve ser registrado em
mídia permanente. Em muitos dos casos, porém, nem todos os sinais coletados
são suficientemente interessantes para que sejam registrados. Para estas
situações, sistemas de filtragem \eng{online} podem ser empregados para
disparar o processo de gravação no momento em que se disponibilizam os sinais
de interesse. Da mesma forma, dependendo do sistema de deteção e aquisição,
estes sistemas podem ter que enfrentar parâmetros de operação bastante
rigorosos. O volume dos dados a ser discriminado poderá ser grande o
suficiente, ou ocorrer em intervalos de tempo curtos o suficiente, que exijam
a distribuição do processamento também no sistema de filtragem.

O processo de discriminação empregado no sistema de filtragem está diretamente
relacionado aos dados sendo adquiridos: os sinais interessantes podem ser
identificáveis através de operações simples sobre sinal coletado ou poderão
exigir um pré-processamento para que seja extraído um conjunto de parâmetros
que simplifiquem o processo de classificação. Nestes casos, é comum empregar
compressão ou compactação dos sinais de entrada do sistema de filtragem para
que se reduza a dimensionalidade do espaço de dados a ser analisado. A
informação a cada análise poderá ainda estar segmentada, o que traz
complexidade adicional ao sistema de processamento de dados. Após a
compactação, compressão ou ambos, é possível que o problema ainda se encontre
em um espaço com dimensionalidade bastante elevada. 

Em situações deste gênero, a classificação exigirá técnicas avançadas para a
identificação dos sinais. Redes Neurais Artificiais vem sendo utilizadas em
muitos problemas nesta área, tanto como forma de compressão dos sinais de
entrada como classificadores, atingindo excelentes níveis de classificação
para igualmente ótimos patamares de desempenho. Redes Neurais são mecanismos
de simples implementação e possuem eficiência comprovada em vários domínios
distintos.

Neste trabalho, apresenta-se uma aplicação onde todos os níveis de
complexidade descritos anteriormente serão exigidos. Neste caso, o sistema de
deteção produzirá eventos em pulsos separados por 25 nanossegundos, a massa de
dados a cada evento estará na ordem de alguns \eng{megabytes} e, a taxa de
eventos de interesse será de apenas alguns para dezenas de milhões.

\section{Motivação}

Experimentos em Física de Altas Energias procuram por confirmações
experimentais dos modelos propostos nos estudos teóricos. Laboratórios deste
domínio da Física contam com um sistema de colisão, que provoca o aparecimento
da física de interesse, associado a complexos sistemas de deteção, que
registram a evolução no tempo de cada evento produzido. Naturalmente
envolvidos no processo de deteção, encontram-se sistemas eletrônicos que
automatizam a busca, registro e análise dos resultados obtidos.

Dada a natureza complexa e rara dos fenômenos estudados em muitos destes
experimentos, a física de interesse está normalmente submersa em uma
gigantesca massa de eventos que representam ora ruído, provocados pelo mal
funcionamento dos sistemas de deteção e colisão, ora eventos ordinários, já
bastante estudados no passado. Em específico, em experimentos que buscam a
confirmação de canais físicos em patamares energéticos elevados, de alguns
gigaelétron-volts para cima, a taxa de eventos que representa canais
desinteressantes contra a de eventos que possam interessar pode estar na faixa
de 10$^6$ para 1. Desta forma, os eventos de interesse aparecem escondidos no
meio de milhões de outros eventos ordinários, ou que representam apenas
ruído. Ademais, para que se consiga apreciar a Física de interesse, milhões de
eventos são gerados por base de tempo para que se colete estatística
suficiente para a comprovação do canal estudado. O volume de dados associados
a cada evento vem aumentando, junto com a ambição dos experimentos. Novos
sistemas de deteção exigem alguns \eng{megabytes} para cada evento registrado,
o que representa uma dificuldade extra na realização destes experimentos.

Para resolver este problema, introduzem-se sistemas eletrônicos de filtragem
que podem selecionar, de forma \eng{online}, os eventos de interesse dos
eventos que representam ruído ou física ordinária. Estes sistemas são por
vezes tão complexos quanto os detetores do experimento e contam com soluções
elegantes para os diversos problemas de transmissão e seleção de
dados. Soluções atuais empregam forte paralelização e técnicas modernas de
processamento de sinais para responder às demandas destes experimentos.

Redes Neurais Artificiais vem sendo empregadas como solução em sistemas de
filtragem em vários experimentos. Dada a forte segmentação dos dados nos
detetores, os sistemas neurais conseguem compactar e extrair as informações
vitais para a discriminação dos dados, mantendo não só a alta qualidade de
classificação como também excelentes níveis de desempenho.

\section{O experimento ATLAS e o bóson de Higgs}

A deteção do bóson de Higgs é um dos grandes expoentes da Física de Altas
Energias atual. Esta partícula, se existir, possui uma massa bastante elevada
(centenas de gigaelétron-volts) e se apresenta como um canal bastante raro e
de difícil reprodução laboratorial. A descoberta desta partícula confirmará
mais uma vez o Modelo Padrão, já bastante testado, inicialmente proposto em
1954.

O CERN, na Suíça, é o local onde está sendo desenvolvido o experimento ATLAS,
que pretende investigar a rara física do bóson de Higgs. O experimento
utilizará colisões próton-próton, numa taxa de 40 milhões por segundo para
conseguir obter alguns destes bósons por dia de operação. As colisões serão
providas pelo Grande Colisionador de Hádrons (do inglês \eng{Large Hadron
Collider}, LHC), que será, quando estiver operacional, o mais potente no
mundo, podendo colidir prótons com 14 TeV no centro de massa.

Uma vez que cada evento no ATLAS consumirá cerca de 1,5 \eng{megabytes} de
espaço em memória, um dos problemas do projeto e construção do experimento
está na articulação de um sistema de filtragem de eventos que seja capaz de
fazer um seleção \eng{online} dos eventos que representem a Física de
interesse.

O Sistema de Filtragem do ATLAS foi inicialmente projetado para operar em três
níveis conectados em cascata com complexidade, qualidade de deteção e tempo de
operação por evento, crescentes. O Segundo Nível de Filtragem (LVL2), em
específico, será constituído de cerca de 1.000 computadores ligados em rede,
processando cada um evento completo aprovado pelo Primeiro Nível. Cada evento
terá em média 10 milissegundos para ser processado.

O LVL2 coordena um conjunto de algoritmos descritos em \eng{software} que
executa a seleção de eventos. Dentre esses, algoritmos de discriminação
elétron/jato têm papel fundamental na eficiência da aquisição de dados, uma
vez que a ocorrência de elétrons pode representar a Física de interesse.

Neste trabalho, registram-se os resultados obtidos para sistemas de
discriminação mais eficientes, baseados em redes neurais artificiais e em um
esquema de compactação do espaço de entrada que explora o perfil de deposição
energética de elétrons e jatos. Para um conjunto de testes de cerca de 600
elétrons e um pouco mais que 2.000 jatos, este sistema apresentou uma
eficiência de classificação de 97\% para elétrons e 95,1\% para
jatos. Comparando-se estes resultados com o algoritmo clássico implementado
atualmente no CERN, observa-se uma melhora de 2 pontos percentuais na
classificação de elétrons contra quase 10 pontos na rejeição de jatos. O
pré-processamento baseado na topologia do evento permite a compactação do
espaço de entrada, originalmente de 58 características extraídas dos dados de
interesse, para apenas 5, mantendo bom desempenho.

Os resultados sugerem que a utilização de ferramentas de processamento de
sinais como a Análise de Componentes Principais e Componentes Independentes
poderá melhorar ainda mais a qualidade da análise executada neste nível de
filtragem. Neste caso, estas ferramentas substituirão o mapeamento topológico
indicado anteriormente para que se encontre um espaço de entrada ainda mais
propício a discriminação elétron/jato.

\section{O que foi feito}

O Sistema de Filtragem do ATLAS representa o trabalho e esforço de muitos
profissionais que se dedicaram em tempo parcial ou integral ao
experimento. Torna-se bastante difícil, em muitos momentos, omitir informações
sobre outras contribuições que, de alguma forma, são importantes à narrativa,
e onde sua supressão prejudicará a compreensão do todo. Os frutos deste
trabalho pertencem a uma cadeia de outros projetos que juntos, compõe a cadeia
completa do sistema de aquisição.

Por estas razões, optou-se pela narrativa exibindo um contínuo, onde as
contribuições deste projeto de cooperação internacional se contextualizam de
uma forma bastante natural e intuitiva. Para guiar o leitor, destacam-se aqui
as contribuições diretas deste trabalho:

\begin{enumerate}
\item Desenvolvimento e aplicação da Unidade Central de Processamento no
LVL2, a L2PU;
\item Desenvolvimento e aplicação da biblioteca global de formatação de dados
do ATLAS (\eng{eformat});
\item Desenvolvimento e aplicação do \eng{Pseudo} Sistema de Leitura (PROS);
\item Desenvolvimento e aplicação de uma biblioteca para rápido acesso e
processamento de dados de calorimetria;
\item Desenvolvimento e aplicação da camada de \eng{software} que combina a
L2PU e a suíte de processamento \eng{offline} Athena;
\item Medidas de desempenho do LVL2 para diferentes cenários, sem a utilização
de algoritmos de discriminação;
\item Desenvolvimento e aplicação de um sistema de mapeamento topológico para
RoI's de objetos ditos eletromagnéticos (e.m.) no LVL2. Este sistema é
utilizado como etapa de pré-processamento para um discriminador neural, que
finalmente classifica elétrons e jatos;
\item Estudos comparativos entre os métodos de compactação topológica
apresentados e o algoritmo atualmente utilizado no CERN para os mesmos fins
(T2Calo); 
\item Estudos de compactação utilizando a informação da relevância para se
distinguirem os canais mais importantes no classificador neural
desenvolvido. Este item inclui estudos comparativos entre a relevância do
discriminador neural baseado no mapeamento topológico e o discriminador
baseado no mapeamento proposto no CERN;
\end{enumerate}

\section{Organização do Texto}

O Capítulo~\ref{chap:introducao} traz uma visão geral da Física de Partículas
atual. Este capítulo inicia-se sumarizando fatos histórios que culminaram na
formulação do Modelo Padrão para a descrição das interações sub-atômicas. Em
seguida, é feita uma revisão das técnicas de aceleração e deteção de
partículas mais comuns em experimentos nesta área. Na
Seção~\ref{sec:calorimetria} descrevem-se detetores conhecidos como
\textit{calorímetros} em maiores detalhes. Aqui, enumeram-se as diversas
partículas de interesse para o trabalho e a forma como interagem com este tipo
de detetor. O capítulo termina introduzindo noções gerais sobre os sistemas de
filtragem que equipam os experimentos modernos.

O Capítulo~\ref{chap:atlas} narra especificidades do experimento ATLAS, que
investigará a Física do bóson de Higgs. O capítulo começa com um breve
histório do CERN, adentrando por detalhes operacionais do experimento em
questão e do colisionador LHC, que gerará as interações próton-próton
necessárias. A Seção~\ref{sec:atlas-calo} traz um resumo das características
operacionais dos Calorímetros do ATLAS e um conjunto de referências para
maiores detalhes.

O Sistema de Filtragem do ATLAS é introduzido no
Capítulo~\ref{chap:trigger}. Este capítulo descreve todos os sub-sistemas que
compõem o Sistema de Filtragem do experimento, desde o Primeiro Nível de
Filtragem até chegar à gravação de um evento selecionado em mídia
permanente. É neste ambiente que são executados os algoritmos de filtragem
abordados neste trabalho. O Segundo Nível de Filtragem é detalhado a partir da
Seção~\ref{sec:lvl2arch}. Resultados obtidos durante simulações do
comportamento desta parte do sistema são apresentados na
Seção~\ref{sec:lvl2work}. O final deste capítulo é dedicado ao ambiente de
processamento de dados Athena e sua interação com o Sistema de Filtragem do
ATLAS.

O Capítulo~\ref{chap:msc} apresenta os resultados obtidos para variações do
sistema de discriminação elétron/jato, utilizando redes neurais artificiais e
um sistema de compactação de informação baseado no perfil de deposição
energética. A partir Seção~\ref{sec:datainput} descrevem-se as características
dos dados disponíveis e os conjuntos utilizados durante a fase inicial deste
trabalho. A Seção~\ref{sec:classical} detalha a técnica de discriminação
elétron/jato empregada atualmente no ATLAS. As seções seguintes descrevem
variações da mesma técnica, que otimizam a separação destes dois tipos de
Física. A partir deste ponto de referência, desenvolvem-se outras técnicas de
discriminação que utilizam redes neurais artificiais. A Seção~\ref{sec:aneis}
descreve o Método dos Anéis, que se utiliza do perfil de interação do objeto
de interesse com os calorímetros para desenvolver uma técnica de separação
mais eficiente. Na Seção~\ref{sec:relevance}, apresenta-se uma técnica de poda
dos dados das regiões de interesse do detetor, de tal forma que se maximize a
capacidade discriminante do Sistema de Filtragem. Estes resultados podem ser
utilizados para a otimização global deste sistema.

O Capítulo~\ref{chap:future} traz uma breve conclusão dos estudos realizados
até o momento presente e apresenta extensões de estudo para este trabalho.

Para a melhor compreensão dos termos utilizados na descrição da geometria dos
objetos estudados neste trabalho, o Apêndice~\ref{ap:coord} introduz o sistema
de coordenadas do ATLAS.

O Apêndice~\ref{chap:published} traz um resumo da produção científica no
período e projetos que integram o corpo deste estudo.

\typeout{ *************** End of file prefacio.tex *************** }
