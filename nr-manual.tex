%% Hello emacs, this is -*- latex -*-
\typeout{ ====================================================================}
\typeout{ This is file nr-manual.tex, created at 22-Nov-2005 }
\typeout{ Maintained by Andre DOS ANJOS <Andre.dos.Anjos@cern.ch> }
\typeout{ ====================================================================}

\chapter{NeuralRinger, projeto e manual de uso}

O pacote \texttt{neuralringer}, atualmente na versão 0.7, inclui um conjunto de
ferramentas, em sua maior parte escrita em C++, constituindo-se de um
laboratório de métodos neurais de deteção usando dados de calorimetria do
experimento ATLAS. Ao longo da vida do projeto, diversas utilidades foram
adicionadas e melhoradas respeitando-se os seguintes critérios:

\begin{itemize}
\item Padronização da codificação e boa qualidade de
documentação~\cite{cpp-codingstd};
\item Desempenho otimizado para emprego no ambiente do Sistema de Filtragem do
ATLAS~\cite{hlt-tdr};
\item Adaptabilidade ao ambiente de análise Athena~\cite{atlas-workbook}.
\end{itemize}

Principalmente devido à escolha da linguagem de codificação padrão por parte da
colaboração do experimento, a grande parte do conjunto de ferramentas foi
escrito em C++. A linguagem também permite a utilização de construções
orientadas à objeto (OO), o que torna mais fácil a manutenção e abstração
necessárias a um projeto de longa vida.

\section{Projeto do sistema}

Para o desenvolvimento do laboratório, compartimentalizou-se o conjunto de
ferramentas em 7 categorias relacionadas a funcionalidade e abstração com
relação a uma das interfaces do sistema. São elas:

\begin{enumerate}
\item \textbf{sys}: Este conjunto de primitivas contém funcionalidade que
define o relatório de erros pelas aplicações do \texttt{neuralringer}, e
abstrações para a utilização de \eng{parsers} XML, obtenção de datas e nomes de
arquivos;
\item \textbf{data}: Contém primitivas para o manuseio de dados em geral,
extração de parâmetros e normalização;
\item \textbf{roiformat}: Contém primitivas que permitem o carregamento e
análise da dados em um formato texto utilizado para transportar dados do
ambiente Athena para aplicações desconectadas deste ambiente;
\item \textbf{rbuild}: Contém primitivas para a configuração, criação e
normalização de anéis, também conhecida como \textit{anelação};
\item \textbf{config}: Primitivas para a configuração de redes neurais
artificiais baseadas em uma descrição XML;
\item \textbf{network}: Primitivas e abstrações para a construção de redes
neurais artificiais;
\item \textbf{lvl1}: Agentes emuladores do Primeiro Nível de Filtragem do
experimento ATLAS~\cite{l1-tdr}.
\end{enumerate}

\subsection{Pacote \texttt{sys}}

\subsection{Pacote \texttt{data}}

\subsection{Pacote \texttt{roiformat}}

\subsection{Pacote \texttt{rbuild}}

\subsection{Pacote \texttt{config}}

\subsection{Pacote \texttt{network}}

\subsection{Pacote \texttt{lvl1}}

\typeout{ *************** End of file nr-manual.tex *************** }
