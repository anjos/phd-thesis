%% Define um ambiente para o resumo, mais compacto e que dá mais texto
\newenvironment{summary}[1]{%
\begin{minipage}{\textwidth}%
\newcommand{\saveparameter}{\baselinestretch}% guarda o valor default
\renewcommand{\baselinestretch}{#1}%
\normalsize}%
{%
\renewcommand{\baselinestretch}{\saveparameter}% restaura o valor default
\end{minipage}%
}

%Deve estar alinhado com o restante do texto. E justificado.

\noindent
Resumo da Tese apresentada à COPPE/UFRJ como parte dos requisitos necessários
para a obtenção do grau de Doutor em Ciências (D.Sc.)

\vspace{1.5cm}

\begin{center}
SISTEMA ONLINE DE FILTRAGEM EM UM AMBIENTE COM ALTA TAXA DE EVENTOS
\vspace{1cm}

André Rabello dos Anjos
\vspace{1cm}

Dezembro/2006
\end{center}
\vspace{1.5cm}

\noindent
Orientador: José Manoel de Seixas
\vspace{1.5cm}

\noindent
Programa: Engenharia Elétrica
\vspace{2cm}

\noindent \begin{summary}{1.2}
\hspace{0.8cm}O experimento ATLAS no CERN, Suíça, contará com um Sistema de
Filtragem que deverá separar a Física ordinária dos eventos que possam
representar decaimentos do raro bóson de Higgs. O Segundo Nível deste Sistema
de Filtragem será constituído de cerca de 1.000 computadores ligados em rede,
processando cada evento aprovado pelo Primeiro Nível em não mais que 10
milissegundos. Neste nível, operará um conjunto de algoritmos descritos em
\eng{software} que executará a seleção de eventos. Dentre estes, algoritmos de
deteção de elétrons têm papel fundamental na eficiência da aquisição de dados,
uma vez que a ocorrência destas partículas pode representar a Física de
interesse. Neste trabalho, apresentamos algoritmos de discriminação mais
eficientes baseados em redes neurais artificiais e um sistema de compactação
de dados que se beneficia do perfil de deposição energético destas partículas
em calorímetros, alcançando uma eficiência de classificação de 97,6\% em
elétrons para apenas 3,2\% de falso-alarme em jatos. Este algoritmo de deteção
é implementado dentro da complexa infraestrutura de \eng{software} do
experimento, podendo ser executado em apenas 125 microssegundos.
\end{summary}

\clearpage

\noindent
Abstract of Thesis presented to COPPE/UFRJ as a partial fulfillment of the
\linebreak requirements for the degree of Doctor of Science (D.Sc.)

\vspace{1.5cm}

\begin{center}
ONLINE TRIGGER SYSTEM FOR A HIGH EVENT RATE ENVIRONMENT 
\vspace{1cm}

André Rabello dos Anjos
\vspace{1cm}

December/2006
\end{center}
\vspace{2cm}

\noindent
Advisor: José Manoel de Seixas
\vspace{2cm}

\noindent
Department: Electrical Enginnering
\vspace{2cm}

\noindent \begin{summary}{1.2}
\hspace{0.8cm}The ATLAS experiment at CERN, Switzerland, will count on a
Trigger System that separates the ordinary physics from that which represents
decays of the rare Higgs boson. The Second Level of such a Trigger System will
be composed 1,000 computers connected by commodity networks, processing each
event approved the the First Level in no more than 10 milliseconds. A set of
algorithms described in software will operate in this filtering level. Between
them, electron detection systems play a fundamental role to the data
acquisition since the existence of these particles can represent interesting
physics. In this work, we present more efficient discrimination algorithms
based on artificial neural networks and a compaction system which benefits
from the energy deposit profiles of these particles in calorimeters, reaching
a classification efficiency of 97.6\% for electrons for a false-alarm of only
3.2\% in jets. This detection algorithm is implemented as part of the
experiment's complex software infraestructure and can be executed in only 125
microseconds.
\end{summary}
