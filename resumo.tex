%% Define um ambiente para o resumo, mais compacto e que dá mais texto
\newenvironment{summary}[1]{%
\begin{minipage}{\linewidth}%
\newcommand{\saveparameter}{\baselinestretch}% guarda o valor default
\renewcommand{\baselinestretch}{#1}%
\normalsize}%
{%
\renewcommand{\baselinestretch}{\saveparameter}% restaura o valor default
\end{minipage}%
}

\noindent
Resumo da Tese apresentada à COPPE/UFRJ como parte dos requisitos necessários
para a obtenção do grau de Doutor em Ciências (D.Sc.)

\vspace{1.5cm}

\begin{center}
SISTEMA ONLINE DE FILTRAGEM EM UM AMBIENTE COM ALTA TAXA DE EVENTOS
\vspace{1cm}

André Rabello dos Anjos
\vspace{1cm}

Dezembro/2006
\end{center}
\vspace{1.5cm}

\noindent
Orientador: José Manoel de Seixas
\vspace{1.5cm}

\noindent
Programa: Engenharia Elétrica
\vspace{2cm}

\begin{summary}{1.2}
\hspace{0.8cm}O experimento ATLAS no CERN, Suíça, contará com um Sistema de
Filtragem que deverá separar a Física ordinária dos eventos que possam
representar decaimentos do raro bóson de Higgs. O Segundo Nível deste Sistema
de Filtragem, em específico, será constituído de cerca de 1.000 computadores
ligados em rede, processando cada um evento completo aprovado pelo Primeiro
Nível. Cada evento terá em média 10 milissegundos para ser processado.  Neste
nível, operará um conjunto de algoritmos descritos em \eng{software} que
executará a seleção de eventos. Dentre esses, algoritmos de discriminação
elétron/jato têm papel fundamental na eficiência da aquisição de dados, uma
vez que a ocorrência de elétrons pode representar a Física de interesse. Neste
trabalho apresentamos resultados obtidos para sistemas de discriminação mais
eficientes, baseados em redes neurais artificiais e um sistema de compactação
de dados que se beneficia do perfil de deposição energético de elétrons e
jatos com calorímetros para alcançar melhor eficiência de classificação. Os
resultados sugerem que a utilização de ferramentas de processamento de sinais
como a Análise de Componentes Principais e Componentes Independentes, poderá
melhorar ainda mais a qualidade da análise executada neste nível de filtragem.
\end{summary}

\clearpage

\noindent
Abstract of Thesis presented to COPPE/UFRJ as a partial fulfillment of the
\linebreak requirements for the degree of Doctor of Science (D.Sc.)

\vspace{1.5cm}

\begin{center}
ONLINE TRIGGER SYSTEM FOR A HIGH EVENT RATE ENVIRONMENT 
\vspace{1cm}

André Rabello dos Anjos
\vspace{1cm}

December/2006
\end{center}
\vspace{2cm}

\noindent
Advisor: José Manoel de Seixas
\vspace{2cm}

\noindent
Department: Electrical Enginnering
\vspace{2cm}

\begin{summary}{1.2}
\hspace{0.8cm}(esta tradução já não condiz com o texto original em português...)
The ATLAS experiment at CERN, Switzerland, will count on a Trigger System that
separates the ordinary Physics from that which represents decays of the rare
Higgs boson. This system was initially designed to operate in 3 levels
cascade-conneted, with higher complexity, detection quality and operation time
per event. The Second Level Trigger, specifically, will be made out of 1,000
commodity computers interconnected forming a network, processing each a full
single event approved by the First Level. Each event shall have an averaged
processing time of 10 miliseconds.

At the second level of this system, operates a set of algorithms described in
software, that will execute the event selection. Among those, electron/jet
discriminators play a fundamental role for the overall efficiency, since
electrons may represent interesting Physics. This work presents results
obtained for more efficient discriminating systems, based on artificial neural
networks. The results suggest that the use of signal processing tools like
Principal Component Analysis or Independent Components may improve even more
the analysis quality executed at this trigger level.
\end{summary}
